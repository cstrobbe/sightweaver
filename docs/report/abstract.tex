% $Id: abstract.tex 269 2003-04-22 14:45:04Z mackers $
\begin{abstract2}

Many tables of data on the web are designed purely for visual presentation
and do not contain semantic information about the structure and content of
that table. This presents problems for visually-disabled people who use the
Internet with adaptive technology such as screen readers.

This paper presents a design and implementation of a tool to analyse and repair
data tables in order for them to adhere to web accessibility standards. In
doing so, screen readers will be able to assist their users to understand
tables in an equivalent manner to non-visually disabled users.

This report looks at the motivation behind creating accessible websites and 
argues the benefits to disabled and non-disabled people alike. An overview
shows how visually disabled people use the web and various accessibility 
standards and legislation is studied.

A suitable design and implementation for a table repair tool is presented, as
well as the issues encountered while developing the tool. An analysis of the
system appraises the tool against accessibility standards and studies
the repaired tables using actual screen reader programs.

Finally, the conclusion evaluates the tool against the requirements and 
presents areas for further investigation.

\end{abstract2}
