% $Id: introduction.tex 269 2003-04-22 14:45:04Z mackers $

\begin{quotation}

The true reason to design for accessibility is greed. Quite simply, I want it
all, and so should you. Give us everything you've got. Give us everything there
is to give.  \cite{joeclark}

\end{quotation}

It is estimated that in the United States alone 40.8 million individuals has
some sort of disability, and 27.3 million of those has a severe disability
\cite{webaim}. This represents some 10-20\% of the population, and this figure
is mirrored in most countries. As is shown in the next chapter, this represents
a significant portion of web traffic and should not be ignored.

The vast majority of websites currently on the web are designed solely with
sighted, non-disabled visitors in mind. Disabled visitors using assistive
technology\footnote{Assistive or adaptive technology can be defined as some
hardware or software that eliminates barriers to using a computer.} may or may
not have complete access to these sites. Because no non-visual information has
been provided, large sections containing images, multimedia or badly marked-up
HTML remain no-go areas.

The greatest shame is that, with a little extra work, website authors and
designers can greatly improve access to their site for all types of visitors.
The reasons that this isn't currently done can be attributed both to ignorance and
to apathy.

What \emph{can} be done, however, is to provide assistance to web publishers in
creating accessible web sites. One type of assistance is the promotion of
accessibility standards and guidelines in the form of documentation and books, 
of which plenty of available. Through these, web designers can 
educate themselves in the technical aspects of website accessibility and can develop
with this in mind with little extra effort. 

However, some aspects of website accessibility can be confusing and difficult
for the reluctant web designer, and impossible for the average content manager.
In an attempt to obviate some of this difficulty, a wide variety of tools
has been developed, ranging from simple accessibility checkers to elaborate
authoring environments and plug-ins. Some of these are discussed in the 
next chapter.

This report details the design and implementation of one such tool, which 
focuses on one particular aspect of accessibility; tables. 

\section{Objectives}

The primary objectives of this report are outlined below:

\begin{itemize}

\item To present the motivation behind the accessibility movement, both from a
social and legal perspective. A background of the demography of disabled web users
will be provided.

\item To explain the accessibility standards and legislation that exists today,
and the problem with existing websites.

\item To outline existing technologies, both assistive technologies and 
accessibility tools.

\item To propose a solution to one aspect of accessibility; tables. 

\item To successfully design and implement a tool which will assist authors
in creating fully accessible HTML tables.

\item To test the output of the tool with accessibility and legal standards.

\item To present a brief discussion on future developments in this area and 
how the tool could be improved.

\end{itemize}

\section{Structure of this Report}

There follows a brief outline of the structure of this report.

\begin{itemize}

\item This chapter has introduced the main concepts and the objectives of the report.

\item Chapter 2 will expand on the basic concepts introduced in the previous
chapter.  It will present some statistics on the numbers of disabled visitors
to websites, and present arguments as to why all websites should be fully
accessible, from a socially responsible and legal point of view. Important auxiliary
benefits will also be listed. 

The chapter will introduce the current state of web accessibility standards for
websites in general and tables in particular. It will also include a summary of
assistive technologies and tools and programs related to this report.

\item In chapter 3, the design for the table repair tool will be detailed. The 
user, domain and system requirements will be presented, and a suitable
architecture proposed.

\item The next chapter will show how the design was implemented throughout
the development cycle. A road-map will show how the implementation was planned
and any problems encountered will be identified together with their solutions.

\item Chapter 5 will be an analysis of the design and implementation of the
tool.  The chapter will show the results of thorough testing to accessibility
standards as well as an evaluation of the project's commitment to its
requirements.

\item 
The report will then conclude with a summary of what of achieved in the project
as well as a discussion of possible future work and developments.

\end{itemize}

